\documentclass{llncs}
\usepackage[style=alphabetic,backend=bibtex]{biblatex}
\addbibresource{kwarcpubs.bib}
\addbibresource{extpubs.bib}
\addbibresource{kwarccrossrefs.bib}
\addbibresource{extcrossrefs.bib}

\title{Demo: TGView3D for Immersive Theory Graph Exploration}
\author{Richard Marcus, Michael Kohlhase, Florian Rabe}
\institute{Computer Science, FAU Erlangen-N\"urnberg}

\begin{document}
\maketitle

Representing mathematical knowledge efficiently is a difficult task due to its inherent complexity.
Modular representation formats like OMDoc/MMT~\cite{Kohlhase:OMDoc1.2,RabKoh:WSMSML13} orgainze it into theories with truth-preserving inclusions and interpretations (morphisms).
To help users understand the global and local strucutre of such representations and libaries, we have experimented with visualizing the theory graphs in the TGView~\cite{RupKohMue:fitgv17} system. But the sheer size of theory graph -- they can contain thousands of nodes and an order of magnitude more edges -- complicates the use if no other processing is done before.

To cope with this, we propose an interactive virtual reality theory-graph viewer: TGView3D. This system follows TGView~\cite{RupKohMue:fitgv17}, an already existing 2D-theory-graph viewer. This tool allows switching between several layouts such as a hierarchical or a force-directed one. Furthermore, nodes can be moved manually and different visual highlighting options are available.

Adding the depth as a third dimension to this concept helps to further organize the theory graph, as effectively more space is generated. Additionally, this offers an inherent way of filtering the information. Since it is possible to freely navigate through the 3D-graph, the user can easily focus on the current visible slice.

Porting the system to virtual reality is the logical consequence. The first advantage is the extremely wide field of view. In combination with the headtracking, this allows for a natural exploration of the represented knowledge. Secondly, special controllers simulate functions of the human hand, e.g. grabbing and pointing, which offer intuitive ways of interaction in the three-dimensional world. Last, the immersive experience of virtual reality stronger stimulates brain areas involved in spatial cognition compared to traditional computer graphics. This could possibly help finding certain patterns in the graph.

In order to optimize the 3D-viewer we will tailor current functionalities toward the
requirements of a virtual reality application, create specialized graph layout algorithms
and find smart ways of presenting the relevant pieces of information. By doing this, we
hope to not only create a system that can present theory graphs in a structured and
intuitive way, but also aid mathematicians in discovering new theories from the presented
knowledge.

The TGView3D system is written in the unity3D framework (\url{https://unity3d.com/}) and the code is available at \url{https://github.com/UniFormal/TGView3D}.

\printbibliography
\end{document}

 
%%% Local Variables:
%%% mode: latex
%%% mode: visual-line
%%% fill-column: 5000
%%% TeX-master: t
%%% End:
